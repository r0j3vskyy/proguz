\documentclass[a4paper]{article} 
\usepackage{latexsym}
\usepackage[MeX]{polski}
\usepackage[utf8]{inputenc}% ew. utf8 lub cp1250
\usepackage{textcomp}
\usepackage[english, polish]{babel} % Dodanie języka polskiego do babel
\usepackage{amsmath, amssymb}  % For \mathcal and \mathbb
\usepackage{amsfonts}           % For \mathfrak
\usepackage{enumitem}
% Zdefiniowanie autora i tytułu:
\author{H.~Szczególny}
\title{Minimalizm}
\frenchspacing
\begin{document}
% Wstawianie autora i tytułu do składu:
\maketitle
% Wstawianie spisu treści:
\tableofcontents
\section{Kilka spostrzeżeń na wstępie} % Poprawiono literówkę "spotrzerzeń"
Właśnie zaczyna się mój cudowny artykuł.
\section{Na pożegnanie}
\ldots{} A~tu się on kończy.
%cudzysłów
``Please press the ‘x’ key.’’
% Myślnik
niebiesko{\dywiz}czarny % \dywiz dla krótkiego myślnika
niebiesko--czarny % -- dla średniego myślnika
$-2$%- matematyczny minus
%tab
http://www.rich.edu/\~{}bush \\%--tabulator
http://www.clever.edu/$\sim$demo
\texteuro \ldots\\ 
%streszczenie
\begin{abstract}
Streszczenie streszczenia.\\
\end{abstract}
Pan~Kowalski ucieszył się\\
na jej widok (zob.~Rys.~5).\\
Podoba mi się JAVA\@. A~tobie?\\
Odsyłacz do tego punktu
\label{sec:this} wygląda tak:
\footnote{Zobacz punkt \ref{sec:this}.}
%wyróżnienie
\emph{Patrz punkt}~\ref{sec:this} na
stronie~\pageref{sec:this}.’’
%stopka
\footnote{Przypis}
%lista
\begin{enumerate}
 \item Taka lista:
 \begin{itemize}
 \item[--] wygląda
 \item[--] śmiesznie.
 \end{itemize}
 \item Pamiętaj:
 \begin{description}
 \item[Głupoty] nie staną się
 mądrościami, gdy się je wyliczy.
 \item[Mądrości] można elegancko
 zestawiać w~wyliczeniach.
 \end{description}
\end{enumerate}
%wycentrowanie
\begin{center}
To jest tekst\\wyśrodkowany.
\end{center}
\begin{verbatim*}
gwiazdkowa wersja
otoczenia verbatim
wyróżnia spacje
w tekście
\end{verbatim*}
%tabela
\begin{tabular}{|r|l|} \hline
7C0 & heksadecymalnie \\ \hline
3700 & oktalnie \\ \hline
11111000000 & binarnie \\ \hline \hline
1984 & dziesiętnie \\ \hline
\end{tabular}
\begin{tabular}{|p{4.7cm}|} \hline
Ten akapit jest wewnątrz pudełka. Mamy nadzieję, że uzyskany efekt się podoba.\\ \hline
\end{tabular}\\
\begin{tabular}{c r @{,} l}
Wyrażenie & \multicolumn{2}{c}{Wartość}\\ \hline
$\pi$ & 3 & 1416 \\ \hline
$\pi^{\pi}$ & 36 & 46 \\ \hline
$(\pi^{\pi})^{\pi}$ & 80662 & 7 \\ \hline
\end{tabular}\\
\begin{tabular}{|c|c|c|c|l|} \hline
1 & \multicolumn{4}{c|}{0} \\ \cline{2-5}
1 & 2 & 3 & 4 & 5 \\ \cline{2-4}
1 & 2 & 3 & 4 & 5 \\ \hline
\end{tabular}
\\
Dyr. E.~K.~Tor,\\ Przewodniczący
Zastępcy\\
\date{16 czerwca 1963~r.} 
\\
{\TeX} należy wymawiać jako
$\tau\epsilon\chi$.\\[6pt]
100~m$^{3}$ wody. \\
To płynie z~mojego~$\heartsuit$\\
Jest $-30\,^{\circ}\mathrm{C}$.
Niedługo zacznę nadprzewodzić.\hbox to 5cm{A}
$a$ do kwadratu plus~$b$
do kwadratu równa się~$c$
do kwadratu.\\ Albo, stosując
bardziej matematyczne
podejście: $c^{2}=a^{2}+b^{2}$.\\[6pt]
$\lim_{n \to \infty}
\sum_{k=1}^n \frac{1}{k^2}
= \frac{\pi^2}{6}$\\
\begin{equation}
\forall x \in \mathbf{R}\colon
\qquad x^{2} \geq 0
\end{equation}
%litery greckie
% \\ $\lambda,\xi,\pi,\mu,\Phi,\Omega$
$\overline{m+n}\underline{x+y}$\\[6pt]
$\underbrace{a+b+\cdots+z}_{26}$
\begin{displaymath}
\hat{y}=x^{2}\quad y' = 2x'''
\end{displaymath}
$\sqrt{x} \sqrt{ x^{2}+\sqrt{y}}
 \sqrt[3]{2} \surd[x^2 + y^2]$
 \begin{displaymath}
 \vec a\quad\overrightarrow{AB}\\[6pt]
 \end{displaymath}
 \begin{displaymath}
 \mathbf{X} =
 \left( \begin{array}{ccc}
 x_{11} & x_{12} & \ldots \\
 x_{21} & x_{22} & \ldots \\
 \vdots & \vdots & \ddots
 \end{array} \right)
 \end{displaymath}
$x\equiv a \pmod{b}$
%kroje pisma
\textbf{ABCdef}
\textit{ABCdef}
$\mathcal{ABC}$
$\mathfrak{ABCdef}$
$\textsf{ABC}$
$\mathbb{ABC}$
\\The main Theorem says that $
\mathcal{P}(\lambda) := \lim_{n \to \infty} \mathcal{P}_n(\lambda)
$
 Pn $(\lambda)$ exists and is a holomorphic
 function of $\lambda$ in the domain  $\{ \operatorname{Re} \lambda > 0 \}$ as well as in some other regions
 \\ \textbf{Twierdzenie 1} (Wielkie twierdzenie Ferma) dla liczby naturalnej $n > 2$ nie
 istnieje takie liczby naturalne dodatnie x,y,z które spełniałyby równanie
 $x^n +y^n =z^n$
 Dowód. W rzeczywistości dowód twierdzenia Fermata przeprowadzony przez
 Wilesa ma dosyć długie historie
\\ \textbf{Defnicja.} Liczby naturalne liczby służące podawaniu liczności (trzy osoby,
 zob. liczebnik główny/kardynalny) i ustalania kolejności (trzecia osoba, zob.
 liczebnik porządkowy).

 Taka lista:
 \begin{itemize}
 \item Czekolada
 \item Kawa
\item Mleko
 \end{itemize}
\ Lista zakupów
 \begin{enumerate}
\item Czekolada
\item Kawa
\item Ciasto
  \end{enumerate}
\section*{Spis zakupów}

\begin{enumerate}[label=\arabic*.]
    \item Warzywa
    \begin{enumerate}[label=\alph*)]
        \item Pomidory
        \item Ogórki
        \item Marchew
    \end{enumerate}
    
    \item Owoce
    \begin{enumerate}[label=\Roman*.]
        \item Jabłka
        \item Banany
        \item Winogrona
    \end{enumerate}
    
    \item Nabiał
    \begin{enumerate}[label=\roman*)]
        \item Mleko
        \item Ser
        \item Jogurt
    \end{enumerate}
    
    \item Pieczywo
    \begin{enumerate}[label=\Alph*)]
        \item Chleb
        \item Bułki
        \item Rogale
    \end{enumerate}
\end{enumerate}
\begin{align}
|\psi_{1,2}|^2 &= \left| 1 + \frac{LC \lambda^2}{2} \pm i \sqrt{-LC \lambda^2 - \left( \frac{LC \lambda^2}{2} \right)^2} \right|^2 \nonumber \\
&= \left( 1 + \frac{LC \lambda^2}{2} \right)^2 - LC \lambda^2 - \left( \frac{LC \lambda^2}{2} \right)^2 = 1. \tag{1}
\end{align}

$$\left\{\begin{array}{rcl}
x+y&=&5\\
x-2y&=&8\\
\end{array} \right.$$
\\
$$f(x)=\left\{\begin{array}{rcl}
\frac {x+5}{12}&dla&x>0\\
x^2 + x - 5&dla&x\leq0\\
\end{array} \right.$$
$$A=\left[\begin{array}{ccc}
12&3&-10\\
x&-15&0\\
2.5&-23&12
\end{array}\right]$$
\begin{table}[h!]
\centering
\begin{tabular}{|c|c|c|}
cell1 & cell2 & cell3 \\ 
cell4 & cell5 & cell6 \\ 
cell7 & cell8 & cell9 \\ 
\end{tabular}
\end{table}
\begin{table}[h!]
\centering
\begin{tabular}{|c|c|c|c|}\hline
Numer & Album & Imie i nazwisko & ocena \\  \hline
1 & 11111 & Jan &\cr Kowalski & 5 \\  \hline
cell7 & cell8 & cell9 \\  \hline
\end{tabular}
\end{table}
\end{document}

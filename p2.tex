\documentclass[a4paper, 12pt]{amsart}
%\usepackage{amsmath}
\usepackage[T1]{fontenc}
\author{D.E.}
\title{Wprowadzenie w tryb matematyczny}
\begin{document}
\maketitle
\section{Środowisko trybu matematycznego}
\subsection{Przykłady}
$$x$$
\[y\]
\begin{displaymath} z \end{displaymath}
\begin{equation} w \end{equation}
\begin{equation*} v \end{equation*}
\subsection{Zadanie}
\begin{equation}
\frac{\frac{1}{x+y}-1}{a+b+c}
\end{equation}
\section{Symbole matematyczne}
\subsection{Zadania na wyszukiwanie}
\subsection{Pierwiastki}
\begin{displaymath}
\sqrt{x}, \sqrt{x}+3, \sqrt[3]{\sqrt{x+7}}, \sqrt[3]{\sqrt{x}+7}
\end{displaymath}
\subsection{Litery greckie}
\begin{displaymath}
\alpha, \beta, \gamma, \delta, \epsilon, \zeta, \eta, \theta, \iota
\end{displaymath} \\
\begin{displaymath}
B(x,y) = \frac{\alpha(x)\alpha(y)}{\alpha(x+y)}
\end{displaymath}
\subsection{Indeksy górne i dolne}
\begin{displaymath}
a_{5}, x^{3+y}, A^{i,j,k}_{n+1}, e^{i\pi} = -1
\end{displaymath}
\subsection{Symbole relacji}
\begin{displaymath}
<, \leq, >, \geq, \subset, \subseteq, \supset, \supseteq, \in, \notin, \forall, \exists, \neq
\end{displaymath}
\begin{displaymath}
A = \{1,x\} \subseteq B = \{1,7,x,(b_{i})_{i \in I}\} \not = C = \{1,7, \{x\}, (b_{i})_{i \in I} \}
\end{displaymath}
\subsection{Logika i teoria mnogości}
\begin{displaymath}
\exists, \not\exists, \forall, \neg, \land, \lor
\end{displaymath}
Stałą liczbę a nazywamy granicą ciągu, jeśli $\forall \varepsilon > 0 \exists N$ że $\forall n > N$ spełniony jest warunek $|a_{n}-a| < \varepsilon$.
\subsection{Zbiory liczbowe}
\begin{displaymath}
\mathbb{N}, \mathbb{Z}, \mathbb{Q}, \mathbb{R}, \mathbb{C}
\end{displaymath}
\begin{displaymath}
\mathbb{N} \subset \mathbb{Z} \subset \mathbb{Q} \subset \mathbb{R} \subset \mathbb{C}
\end{displaymath}
\subsection{Funkcje}
\begin{displaymath}
\cos x, \sin x, \lg x
\end{displaymath}
\begin{displaymath}
\cos(2\theta) = \cos^{2}(\theta) - \sin^{2}(\theta)
\end{displaymath}
\subsection{Sumy, iloczyny i całki}
\begin{displaymath}
\sum _{k=1}^{n},
\end{displaymath}
\subsection{Zadanie. Funkcja Eulera}
\end{document}
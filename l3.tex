\documentclass{article}
\usepackage[MeX]{polski}
\usepackage[utf8]{inputenc}
\begin{document}
\title{Lekcja nr 3}
\author{D. E.}

Na pamięć znam tylko jeden angielski wiersz. Ten o~Humptym Dumptym.
\begin{flushleft}
\begin{verse}
Humpty Dumpty sat on a wall:\
Humpty Dumpty had a great fall.\\
All the King's horses and all the King's men\\
Couldn't put Humpty together again.
\end{verse}
\end{flushleft}

\begin{verbatim}
{for (i=1;i<=NF;i++) {l[$i]++;}
END {for (i in l) {print l[i]}
\end{verbatim}

\begin{verbatim*}
gwiazdkowa wersja
otoczenia verbatim
wyróżnia spacje
w tekście
\end{verbatim*}

$a$ do kwadratu plus~$b$ do kwadratu równa się~$c$ do kwadratu. Albo, stosując bardziej matematyczne podejście: $c^{2}=a^{2}+b^{2}$.
\\ \\
{\TeX} należy wymawiać jako $\tau\epsilon\chi$.\\[6pt]
100~m$^{3}$ wody. \\[6pt]
To płynie z~mojego~$\heartsuit$.
\\ \\

\begin{tabular}{c r @{,} l}
Wyrażenie & \multicolumn{2}{c}{Wartość}\\\hline
$\pi$ & 3&1416 \\
$\pi^{\pi}$ & 36&46 \\
$(\pi^{\pi})^{\pi}$ & 80662&7 \\
\end{tabular}
\\ \\
\begin{tabular}{|c|c|c|c|l|}\hline
1 &\multicolumn{4}{c|}{0}\\
\cline{2-5}
1 & 2 & 3 & 4 &5 \\ \cline{2-4}
1 & 2 & 3 & 4 &5 \\ \hline
\end{tabular}
\\
\begin{equation}
\epsilon > 0 \label{eq:eps}
\end{equation}
Ze wzoru (\ref{eq:eps}) otrzymujemy \ldots
\\
$\lim_{n \to \infty}
\sum_{k=1}^n \frac{1}{k^2} = \frac{\pi^2}{6}$
\\
\begin{displaymath}
\lim_{n \to \infty}
\sum_{k=1}^n \frac{1}{k^2} = \frac{\pi^2}{6}
\end{displaymath}
\\
\begin{equation}
\forall x \in \mathbf{R}\colon
\qquad x^{2} \geq 0
\end{equation}

$\lambda, \xi, \pi, \mu, \Phi, \Omega$

$a_{1} x^{2} e^{-\alpha t}
a^{3}_{ij} e^{x^2} \neq {e^x}^2$










\end{document}